	%% This is file `elsarticle-template-1-num.tex',
%%
%% Copyright 2009 Elsevier Ltd
%%
%% This file is part of the 'Elsarticle Bundle'.
%% ---------------------------------------------

\documentclass[a4paper]{article}

%% Use the option review to obtain double line spacing
%% \documentclass[preprint,review,12pt]{elsarticle}

%% Use the options 1p,twocolumn; 3p; 3p,twocolumn; 5p; or 5p,twocolumn
%% for a journal layout:
%% \documentclass[final,1p,times]{elsarticle}
%% \documentclass[final,1p,times,twocolumn]{elsarticle}
%% \documentclass[final,3p,times]{elsarticle}
%% \documentclass[final,3p,times,twocolumn]{elsarticle}
%% \documentclass[final,5p,times]{elsarticle}
%% \documentclass[final,5p,times,twocolumn]{elsarticle}

%% The graphicx package provides the includegraphics command.
\usepackage{graphicx}
\usepackage{import}
\usepackage{xifthen}
\usepackage{pdfpages}
\usepackage{transparent}
\usepackage{cancel}
%% The amssymb package provides various useful mathematical symbols
\usepackage{amssymb}
\usepackage{amsmath}
\usepackage{amsthm}
\usepackage{ mathrsfs }
\usepackage{ dsfont }
\usepackage{tikz-cd}
\usepackage[a4paper,top=3cm,bottom=2cm,left=3cm,right=3cm,marginparwidth=1.75cm]{geometry}
\usepackage{float}
%% The lineno packages adds line numbers. Start line numbering with
%% \begin{linenumbers}, end it with \end{linenumbers}. Or switch it on
%% for the whole article with \linenumbers after \end{frontmatter}.
\usepackage{lineno}
\usepackage{hyperref}
\usepackage{enumitem}
%% natbib.sty is loaded by default. However, natbib options can be
%% provided with \biboptions{...} command. Following options are
%% valid:

%%   round  -  round parentheses are used (default)
%%   square -  square brackets are used   [option]
%%   curly  -  curly braces are used      {option}
%%   angle  -  angle brackets are used    <option>
%%   semicolon  -  multiple citations separated by semi-colon
%%   colon  - same as semicolon, an earlier confusion
%%   comma  -  separated by comma
%%   numbers-  selects numerical citations
%%   super  -  numerical citations as superscripts
%%   sort   -  sorts multiple citations according to order in ref. list
%%   sort&compress   -  like sort, but also compresses numerical citations
%%   compress - compresses without sorting
%%
%% \biboptions{comma,round}

% \biboptions{}
\newtheorem{name}{Printed output}
\newtheorem{mydef}{Definition}
\newtheoremstyle{break}% name
  {}%         Space above, empty = `usual value'
  {}%         Space below
  {}%         Body font
  {}%         Indent amount (empty = no indent, \parindent = para indent)
  {\bfseries}% Thm head font
  {.}%        Punctuation after thm head
  {\newline}% Space after thm head: \newline = linebreak
  {}%         Thm head spec

\theoremstyle{plain} %italics
\newtheorem{question}{Question}

\newtheorem{theorem}{Theorem}[section]
%\numberwithin{theorem}{subsection}

\newtheorem*{lemma*}{Lemma}
\newtheorem{lemma}[theorem]{Lemma}
%\numberwithin{lemma}{subsection}

\newtheorem{corollary}[theorem]{Corollary}
%\numberwithin{corollary}{subsection}

\newtheorem{proposition}[theorem]{Proposition}
%\numberwithin{proposition}{subsection}

\theoremstyle{definition} %non italics

\newtheorem{definition}[theorem]{Definition}
%\numberwithin{definition}{subsection}

\newtheorem*{note*}{Note}

\newtheorem*{claim}{Claim}

\newtheorem*{remark}{Remark}

\newtheorem*{example}{Example}

\numberwithin{equation}{subsection}

\newcommand{\incfig}[1]{%
    \def\svgwidth{\columnwidth}
    \import{./figures/}{#1.pdf_tex}
}
\newcommand{\hodge}{{*}}
\newcommand{\tang}{\operatorname{\textbf{\textup{t}}}}
\newcommand{\norm}{\operatorname{\textbf{\textup{n}}}}
%mathbb short cut
\newcommand{\R}{\mathbb{R}}
\newcommand{\bP}{\mathbb{bP}}
\newcommand{\N}{\mathbb{N}}
\newcommand{\Q}{\mathbb{Q}}
\newcommand{\A}{\mathbb{A}}
\newcommand{\bS}{\mathbb{S}}
\newcommand{\Z}{\mathbb{Z}}
\newcommand{\F}{\mathbb{F}}
\newcommand{\B}{\mathbb{B}}
\newcommand{\T}{\mathbb{T}}
\newcommand{\K}{\mathbb{K}}
\newcommand{\C}{\mathbb{C}}
\newcommand{\D}{\mathbb{D}}

\usepackage{comment}
%bm short cut
\newcommand{\bmv}{\bm{v}}
\newcommand{\bmu}{\bm{u}}
\newcommand{\bmw}{\bm{w}}
\newcommand{\bmx}{\bm{x}}

%mathcal shortcut
\newcommand{\cF}{\mathcal{F}}
\newcommand{\cA}{\mathcal{A}}
\newcommand{\cM}{\mathcal{M}}
\newcommand{\cU}{\mathcal{U}}
\newcommand{\cB}{\mathcal{B}}
\newcommand{\cD}{\mathcal{D}}
\newcommand{\cL}{\mathcal{L}}
\newcommand{\cS}{\mathcal{S}}
\newcommand{\cG}{\mathcal{G}}
\newcommand{\cT}{\mathcal{T}}
\newcommand{\cC}{\mathcal{C}}
\newcommand{\cH}{\mathcal{H}}
\newcommand{\cO}{\mathcal{O}}
\newcommand{\cJ}{\mathcal{J}}

%mathfraktur shortcut
\newcommand{\frp}{\mathfrak{p}}
\newcommand{\frq}{\mathfrak{q}}
\newcommand{\frm}{\mathfrak{m}}
\newcommand{\frP}{\mathfrak{P}}

%mathscr shortcut
\newcommand{\msF}{\mathscr{F}}
\newcommand{\msA}{\mathscr{A}}
\newcommand{\msM}{\mathscr{M}}
\newcommand{\msE}{\mathscr{E}}
\newcommand{\msU}{\mathscr{U}}
\newcommand{\msB}{\mathscr{B}}
\newcommand{\msD}{\mathscr{D}}
\newcommand{\msL}{\mathscr{L}}
\newcommand{\msS}{\mathscr{S}}
\newcommand{\msG}{\mathscr{G}}
\newcommand{\msT}{\mathscr{T}}
\newcommand{\msC}{\mathscr{C}}
\newcommand{\msH}{\mathscr{H}}
\newcommand{\msO}{\mathscr{O}}
\newcommand{\msJ}{\mathscr{J}}
\newcommand{\msP}{\mathscr{P}}


%set notation shortcut
\newcommand{\vect}[1]{\underline{\textbf{#1}}}
\newcommand{\llangle}{\left\langle}
\newcommand{\rrangle}{\right\rangle}
\newcommand{\llbrace}{\left\lbrace}
\newcommand{\rrbrace}{\right\rbrace}
\newcommand{\setof}[1]{\left\lbrace #1 \right\rbrace}


\newcommand{\cFA}{\cF(\cA)}
\newcommand{\sigmaA}{\sigma(\cA)}
\newcommand{\del}[2]{\frac{\partial #1}{\partial #2}}


\newcommand{\innerproductb}[2]{\llangle\bm{#1},\bm{#2}\rrangle}
\newcommand{\innerproduct}[2]{\llangle{#1},{#2}\rrangle}

\newcommand{\cl}{\operatorname{cl}}
\newcommand{\coker}{\operatorname{coker}}
\newcommand{\re}{\operatorname{Re}}
\newcommand{\im}{\operatorname{Im}}
\newcommand{\dist}{\operatorname{dist}}
\newcommand{\Span}{\operatorname{Span}}
\newcommand{\Supp}{\operatorname{Supp}}
\newcommand{\id}{\operatorname{id}}
\newcommand{\sh}{\operatorname{sh}}
\newcommand{\var}{\operatorname{Var}}
\newcommand{\Proj}{\operatorname{Proj}}
\newcommand{\Ker}{\operatorname{Ker}}
\newcommand{\Mor}{\operatorname{Mor}}
\newcommand{\supp}{\operatorname{supp}}
\newcommand{\Ran}{\operatorname{Ran}}
\newcommand{\Res}{\operatorname{Res}}
\newcommand{\Mod}{\operatorname{Mod}}
\newcommand{\res}{\operatorname{res}}
\newcommand{\Frac}{\operatorname{Frac}}
\newcommand{\End}{\operatorname{End}}
\newcommand{\Sym}{\operatorname{Sym}}
\newcommand{\Spec}{\operatorname{Spec}}
\newcommand*{\sheafhom}{\mathcal{H}\kern -.5pt \it{ om}}
\newcommand*{\sheafspec}{\mathcal{S}\kern -.5pt \it{ pec}}
\newcommand*{\sheafproj}{{\mathcal{P}\kern -.5pt \it{ roj}}}
\newcommand{\Div}{\operatorname{div}}
\newcommand{\Aut}{\operatorname{Aut}}
\newcommand{\Hom}{\operatorname{Hom}}
\newcommand{\Qcoh}{\operatorname{Qcoh}}
\newcommand{\PGL}{\operatorname{PGL}}
\newcommand{\Op}{\operatorname{Op}}
\newcommand{\WF}{\operatorname{WF}}
\newcommand{\Char}{\operatorname{Char}}
\newcommand{\Ell}{\operatorname{Ell}}
\newcommand{\RE}{\operatorname{Re}}
\newcommand{\IM}{\operatorname{Im}}


\begin{document}

%% Title, authors and addresses

\title{\textbf{LaTeX in VSCode Example}}

%% use the tnoteref command within \title for footnotes;
%% use the tnotetext command for the associated footnote;
%% use the fnref command within \author or \address for footnotes;
%% use the fntext command for the associated footnote;
%% use the corref command within \author for corresponding author footnotes;
%% use the cortext command for the associated footnote;
%% use the ead command for the email address,
%% and the form \ead[url] for the home page:
%%
%% \title{Title\tnoteref{label1}}
%% \tnotetext[label1]{}
%% \author{Name\corref{cor1}\fnref{label2}}
%% \ead{email address}
%% \ead[url]{home page}
%% \fntext[label2]{}
%% \cortext[cor1]{}
%% \address{Address\fnref{label3}}
%% \fntext[label3]{}


%% use optional labels to link authors explicitly to addresses:
%% \author[label1,label2]{<author name>}
%% \address[label1]{<address>}
%% \address[label2]{<address>}

\author{\textbf{Jeffrey Thompson}}
\maketitle
\begin{abstract}
Abstract
\end{abstract}
\tableofcontents
\newpage
\section{Averages}
Let $x_1,\dots, x_n$ be numeric recorded data values.

\begin{definition}
\textbf{Average (mean)}:
$$\overline{x} = \frac{x_1 + \dots + x_n}{n}= \frac{\Sigma_{i=1}^nx_i}{n}= \frac{\operatorname{SUM}([x])}{\operatorname{COUNT}([x])} = \operatorname{AVG}([x])$$
\end{definition}

Note that 

$$\overline{x}= \frac{1}{n}x_1 + \dots + \frac{1}{n}x_n$$

A weighted average is an average which in which some of the data is given higher or lower priority. More rigorously, if we have weights $w_1, \dots, w_n$ where each $w_i$ is between $0$ and $1$, and $w_1 + \dots + w_n = 1$, Then

\begin{definition}
\textbf{Weighted Average}:
$$\overline{x} = w_1x_1 + \dots + w_nx_n= \Sigma_{i=1}^nw_ix_i= \operatorname{SUM}([w]\times[x])$$
\end{definition}
Note that the mean can be viewed as a weighted average with each weight being $\frac{1}{n}$. 

What weights should we choose? 
\subsection{Our favourite weighted average}
An important use-case of weighted averages is the ``correct'' way to take averages of averages (or percentages). If we \textbf{have} two \textbf{averages already} $\overline{x}$ and $\overline{y}$, and the \textbf{number of records used to compute those averages}, $n$ and $m$, as in
$$\overline{x} =  \frac{x_1 + \dots + x_n}{n}$$
$$\overline{y} =  \frac{y_1 + \dots + y_m}{m}$$
Then we can compute the average of both via the formula:
\begin{example}
    \textbf{Weighted Average of averages}

    $$ \overline{z} = \frac{n\overline{x}+m\overline{y}}{n+m} = \frac{\operatorname{SUM}([\text{Number of records}]\times[\text{Averages}])}{\operatorname{SUM}([\text{Number of records}])}$$
Where $\overline{z}$ is the average for the row-level numeric data values for $x$ and $y$ combined (union).

Note that this is the same as

$$\overline{z}= \frac{n}{n+m}\overline{x}+ \frac{m}{n+m}\overline{y}$$

Where you can notice that $\frac{n}{n+m}\times100$ and $\frac{m}{n+m}\times100$ are the percentages of the total number of records!
\end{example}
Why this formula? We know that 
$$ \frac{n\overline{x}+m\overline{y}}{n+m} =  \frac{\cancel{n}\frac{x_1 + \dots + x_n}{\cancel{n}}+\cancel{m}\frac{y_1 + \dots + y_m}{\cancel{m}}}{n+m} = \frac{x_1 + \dots + x_n + y_1 + \dots + y_m}{n+m} $$

Which is exactly the formula for the average of the combined data field of $x$ and $y$! Therefore, this is the correct way to take averages of averages. 

This is \textbf{not} equal to 
$$\text{naive average} = \frac{\overline{x}+ \overline{y}}{2}$$

Note that similar logic can be applied for when you have more than two averages.

\newpage

\section{Quadratic Formula via Sympy Example}
Start: $$ a x^{2} + b x + c = 0 $$

Start (simplified): $$ a x^{2} + b x + c = 0 $$

Divide by a: $$ \frac{a x^{2} + b x + c}{a} = \frac{0}{a} $$

Divide by a (simplified): $$ \frac{a x^{2} + b x + c}{a} = 0 $$

Move constant to RHS: $$ - \frac{c}{a} + \frac{a x^{2} + b x + c}{a} = - \frac{c}{a} + \frac{0}{a} $$

Move constant to RHS (simplified): $$ \frac{x \left(a x + b\right)}{a} = - \frac{c}{a} $$

Add $(\frac{b}{2a})^2$ to both sides: $$ \left(- \frac{c}{a} + \frac{a x^{2} + b x + c}{a}\right) + \frac{b^{2}}{4 a^{2}} = \left(- \frac{c}{a} + \frac{0}{a}\right) + \frac{b^{2}}{4 a^{2}} $$

Add $(\frac{b}{2a})^2$ to both sides (simplified): $$ \frac{a x \left(a x + b\right) + \frac{b^{2}}{4}}{a^{2}} = \frac{- a c + \frac{b^{2}}{4}}{a^{2}} $$

Perfect square form (unsimplified RHS): $$ \left(x + \frac{b}{2 a}\right)^{2} = \left(- \frac{c}{a} + \frac{0}{a}\right) + \frac{b^{2}}{4 a^{2}} $$

Perfect square form (unsimplified RHS) (simplified): $$ \left(x + \frac{b}{2 a}\right)^{2} = \frac{- a c + \frac{b^{2}}{4}}{a^{2}} $$

Perfect square form (simplified RHS): $$ \left(x + \frac{b}{2 a}\right)^{2} = \frac{- a c + \frac{b^{2}}{4}}{a^{2}} $$

Perfect square form (simplified RHS) (simplified): $$ \left(x + \frac{b}{2 a}\right)^{2} = \frac{- a c + \frac{b^{2}}{4}}{a^{2}} $$

Square root (plus branch): $$ x + \frac{b}{2 a} = \frac{\sqrt{- 4 a c + b^{2}}}{2 a} $$

Square root (plus branch) (simplified): $$ x + \frac{b}{2 a} = \frac{\sqrt{- 4 a c + b^{2}}}{2 a} $$

Square root (minus branch): $$ x + \frac{b}{2 a} = - \frac{\sqrt{- 4 a c + b^{2}}}{2 a} $$

Square root (minus branch) (simplified): $$ x + \frac{b}{2 a} = - \frac{\sqrt{- 4 a c + b^{2}}}{2 a} $$

Solve for x (plus): $$ x = \frac{- b + \sqrt{- 4 a c + b^{2}}}{2 a} $$

Solve for x (plus) (simplified): $$ x = \frac{- b + \sqrt{- 4 a c + b^{2}}}{2 a} $$

Solve for x (minus): $$ x = \frac{- b - \sqrt{- 4 a c + b^{2}}}{2 a} $$

Solve for x (minus) (simplified): $$ x = \frac{- b - \sqrt{- 4 a c + b^{2}}}{2 a} $$

Quadratic formula: $$ x = \frac{-\,b \pm \sqrt{b^{2}-4ac}}{2a} $$



\end{document}


%%
%% End of file `elsarticle-template-1-num.tex'.
